\abstract{The Parallel Distributed Processing (PDP) approach to cognition assumes that active mental representations are distributed patterns of neural activation, with each neural population contributing to many different representations and each representation involving many populations. Distributed representations of this kind provide the basis for influential models in many different domains of cognition. Until recently, however, the PDP view of representation has not strongly influenced functional brain imaging, which has, for historical and methodological reasons, tended to adopt modular assumptions about how the brain encodes information. New methods for multivoxel pattern analysis (MVPA) relax these but rely on other important assumptions about the nature of the underlying representations that sometimes go unspecified. In this paper we consider how analysis of functional brain imaging data is best approached if the PDP assumptions about representation are valid. To this end, we compare and contrast five different data analysis methods in their ability to discover distributed representations in artificial data generated from a simple neural network model. We illustrate how the assumptions underlying each method lead to bias in the nature of signal detected, and show that some methods will fare better at discovering the relevant information if the PDP assumptions are valid. We then compare performance of the different methods in analysis of one well-known publicly available fMRI dataset, with results suggesting that, at least in this case, the core representations do indeed have the properties assumed by PDP. The results suggest a general strategy for fMRI data analysis that may provide a fuller picture of how information is represented and coded in neural activity, and may allow for a better connection between mechanistic models of cognitive function and data arising from functional brain imaging.}
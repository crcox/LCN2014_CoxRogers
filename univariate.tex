\subsection{Univariate contrast analysis}
\subsubsection{Concept and assumptions} The univariate contrast analysis is the standard method for interrogating fMRI data. Its goal is to identify regions of cortex that, across subjects, exhibit systematically different mean BOLD responses to two (or more) different kinds of cognitive events. Typically the BOLD signal is spatially smoothed, so that the raw response at each voxel is replaced with a weighted average of the responses from anatomically neighboring voxels. The smoothed time-series is then modeled independently at each voxel for each subject using a deconvolution procedure. This yields a beta coefficient for each experiment condition at each voxel indicating how well the measured BOLD signal matches the response expected if the activation of neurons within the voxel varies systematically with the experiment condition. The beta coefficients for each subject are projected into a common anatomical reference space, and univariate statistical tests are computed at each voxel independently to assess whether the coefficients differ reliably in the two experimental conditions across subjects. Voxels that show significantly different responses across subjects are viewed as important for coding the representation of interest. 

A major challenge for the approach lies in establishing a meaningful criterion of significance in the context of tens or even hundreds of thousands of individual statistical tests. To avoid both false-positives and punishing corrections for multiple comparisons, it is common to seek ways of reducing the number of tests performed. Several different methods have been employed, but all rely on the idea that the representations of interest can be localized to particular cortical regions, and that the responses of voxels within a functional region will be largely similar. With these assumptions, the number of tests can be reduced by (1) conducting regions-of-interest analyses, where the responses of voxels within a ROI are averaged and the test is performed on the result mean response, (2)applying cluster-thresholding, where tests are only performed on clusters of {\em n} anatomically contiguous voxels all showing a similar response across subjects, or (3) applying a topographic control of the false-discovery rate. 
 
The univariate contrast method thus favors the discovery of clusters of anatomically neighboring voxels located in similar regions across individuals and showing similar response profiles across experimental conditions. 

From this brief description we can see that the method relies on five assumptions about the nature of the neuro-cognitive representations, summarized in the ``univariate'' column of Table \ref{tab.assumptions}.
%
%\begin{enumerate}
%\item Localization within individuals: Representations are anatomically localized within individuals, so that neighboring voxels all contribute to the same representation.
%
%\item Consistency of coding within individual representations: Within individuals, voxels contributing to a given representation encode the same information in the same way---for instance, always showing greater activation for one condition relative to another.
%
%\item Localization across individuals: A given representation is encoded in anatomically similar regions across individuals.
%
%\item Consistency of coding across individuals: A voxel at a given anatomical location will code the same information in the same way across individuals.
%
%\item Independence of representational units: The interpretation of a voxel's activation does not depend on the states of other voxels.
%\end{enumerate}
%
%Assumptions 1 and 2 license spatial smoothing of the raw BOLD signal, a data processing step that will only improve signal if these are both valid. Assumptions 3--5 license the logic of the general approach---that of seeking single voxels in similar anatomical locations across individuals that exhibit a similar pattern of contrasts across experimental conditions. The five assumptions together license the common methods for dealing with the vast amount of data by reducing the number of statistical tests performed. 
%
%With these assumptions, what does the univariate contrast approach discover in the model?

\subsubsection{Implementation}
The activity at each unit was modeled simultaneously for all subjects in a mixed effects model that treated subject as a random factor \cite{chen_linear_2013, friston_mixed-effects_2005} using the lme4 package in R \cite{bates_linear_2013}. Each model contains a single regressor, coding whether each item is an example of category A or B. The coefficients obtained from the mixed effects model were tested for significance using the Kenward-Roger approximation for the degrees of freedom \cite{kenward_small_1997} and a standard F-test, numerator degrees of freedom = 1, denominator degrees of freedom = 9. The results are directly analogous to a repeated-measures ANOVA. The criterion for significance, alpha, is corrected to control the false discovery rate at q<0.05. The analysis was conducted for both the anatomically localized and the dispersed model. In both cases, the data were spatially smoothed, taking a weighted average over a three unit window, where the center unit was weighted about twice as much as the two flanking units.

\subsubsection{Results} 

\begin{center}
\textbf{---Figure \ref{fig.univariate} about here---}
\end{center}

\begin{figure}
\centering
\includegraphics[width=0.95\textwidth]{figures/figure4.eps}
\caption{Results from the univariate analysis of simulated data. Bar height indicates the absolute value of the $t$ statistic for the unit-wise contrast between conditions at the group level. Colored bars indicate units showing significant differences with p-values corrected to control the false discovery rate at $q<0.05$. The red-blue scale indicates the direction of the contrast effect across model subjects, with red indicating units consistently showing greater activation for A items.}
\label{fig.univariate} 
\end{figure}

Figure \ref{fig.univariate} shows the results of applying the univariate method to the localized (left) and dispersed (right) models. In these plots, each bar corresponds to a single unit in the model. The bars are ordered according to their functional role in the network, as indicated by the X-axis labels. Colored bars indicate units showing statistically significant differences in mean activation across model individuals, while grey bars indicate units that did not show significant differences. Among the colored bars, red indicates units where activation was systematically higher for domain A across models, and blue indicates units where activation was systematically higher for domain B. Note that, in the anatomically dispersed plot (right), the units are shown in their standard functional location for ease of interpretation.

In both localized and dispersed cases, the univariate contrast method identifies systematic I/O units as important for representing the A/B distinction, and correctly indicates that different subsets of input units code this information differently (some responding more to A than B and others showing the opposite pattern). Note that these are the units for which the five univariate assumptions about representation are all valid. In both localized and dispersed cases, however, the analysis completely misses the systematic hidden units, even though these jointly encode a cleaner representation of the A/B domain structure. The failure arises because, in both cases, the univariate assumptions are invalid. When hidden units are localized, assumptions 2 and 4 are violated: the way individual units encode information can vary across SH units in the same model individual, and across individuals at the same anatomical location. When the units are anatomically dispersed, assumption 3 is also invalid: the representation is coded in different anatomical locations across individuals. Because of these departures from the statistical assumptions, the mean activation of a unit at a given anatomical location across individuals does not differ reliably for SH units, even though these do reliably encode the domain distinction in each individual. These results are summarized in the column labeled ``univariate'' in Table \ref{tab.modelresults}

%In summary, we get the following answers to the core questions:
%
%\begin{enumerate}
%\item {\bf Does the method reliably identify the systematic I/O units?} Yes.
%\item {\bf Does the method reliably identify the systematic hidden units?} No.
%\item {\bf Does the method indicate how the information of interest is coded across identified units?} Yes.
%\item {\bf Do method results depend on anatomical localization of signal-carrying units?} No---hidden units are not discovered regardless of anatomical layout.
%\end{enumerate}


\documentclass{beamer}
\usepackage{biblatex}
\addbibresource{/Users/lcnllab/Documents/MASTERBIB/zotero.bib}

\usepackage{multirow}
\usepackage{amssymb}
\usepackage{etoolbox}
\DeclareCiteCommand{\footpartcite}[\mkbibfootnote]
  {\usebibmacro{prenote}}
  {%\usebibmacro{citeindex}\mkbibbrackets{\usebibmacro{cite}}\setunit{\addnbspace}
   \printnames{labelname}\setunit{\labelnamepunct}
%   \printfield[citetitle]{title}%
   \newunit
   \printfield{year}}
  {\addsemicolon\space}
  {\usebibmacro{postnote}}
  
\usetheme{CambridgeUS}
\setbeamertemplate{items}[default]
\setbeamercolor*{item}{fg=darkred}
%\addtobeamertemplate{block begin}{}{\setlength{\parskip}{35pt plus 10pt minus 10pt}}
\setlength{\parskip}{16pt plus 1pt minus 1pt}

\title[SOS LASSO]{SOS LASSO: A new method for finding distributed representations in fMRI data.}
\author{Chris Cox}
\institute{University of Wisconsin, Madison}
\date{\today}

\newcommand {\framedgraphic}[2] {
    \begin{frame}{#1}
        \begin{center}
            \includegraphics[width=\textwidth,height=0.8\textheight,keepaspectratio]{#2}
        \end{center}
    \end{frame}
}

\begin{document}

%%%
%% TITLE PAGE 
%%%
\begin{frame}
\titlepage
\end{frame}

%%%
%% OUTLINE
%%%
%\begin{frame}
%\frametitle{Outline}
%\tableofcontents
%\end{frame}

%%%
%% INTRODUCTION
%%%
\section{Introduction}

\begin{frame}
\frametitle{Background}
\begin{itemize}
\item PDP models have been gainfully used to develop theory regarding variety of behaviors cognitive phenomena. 
\item The way in which information is encoded in these models---in the form of {\bf distributed representations}---is responsible for why the models behave as they do.
\item However, there is limited neural evidence for distributed representation in the brain.
\item {\bf This may be largely for methodological reasons.} The assumptions inherent in many neuroimaging methods make them ill-suited to discovering distributed representations in neural data.
\end{itemize}
\end{frame}

\section{Method}
\begin{frame}
\frametitle{Method}
\begin{enumerate}
\item We generated data 
\end{enumerate}
\end{frame}

%%%
%% WHY DIST REPS ARE HARD
%%%
\begin{frame}
\frametitle{Challenges of distributed representation}
\begin{enumerate}
  \item The behavior of a given cortical subregion (i.e., voxel or voxel cluster) can vary substantially across individuals even if different individuals encode the same representational structure across the same general regions.
  \pause
  \item Activation of individual units may not be interpretable independent of other units.
  \pause
  \item The functional model architecture may not map transparently onto anatomical structure in the brain.
  \pause
  \item The network of interest in any given study co-exists in the brain with many other networks, all subserving other functions that may not be of interest.
\end{enumerate}
\end{frame}

%%%
%% DATA EXAMPLE
%%%
\framedgraphic{A small glimpse of the problem}{data_demo_1.pdf}
\framedgraphic{A small glimpse of the problem}{data_demo_2.pdf}
\framedgraphic{A small glimpse of the problem}{data_demo_3.pdf}
\framedgraphic{A small glimpse of the problem}{data_demo_4.pdf}


%%%
%% THE PLAN
%%%
\begin{frame}
\frametitle{Method}
\begin{enumerate}
\item Generate data that instantiates these challenges. Some units in each dataset adhere to standard assumptions about how neural units behave (neighboring units activate in similar ways; units activate similarly across subjects), and other units that participate in distributed representations that violate these assumptions.
\item Analyze these datasets with a set of methods that make different assumptions about the underlying signal. The different outcomes follow directly from these assumptions, and what structure in the data the method is sensitive to.
\item Through this analysis, the relative strengths and weaknesses for each method are illustrated.
\end{enumerate}
\end{frame}

\begin{frame}
\frametitle{Overview}
\begin{tabular}{l|c|c|c|c|}
 & \multicolumn{2}{c|}{Assumes same ...} & \multicolumn{2}{c|}{ Provides ...} \\
 & Location & Encoding  & Unit info & Importance \\
Univariate & \checkmark &  \checkmark & \checkmark & \checkmark\\
Searchlight & & \checkmark &  & \checkmark \\
Ridge & & & \checkmark & \\
LASSO & & & \checkmark & \checkmark \\
SOS LASSO & \checkmark & & \checkmark & \checkmark \\
\end{tabular}

%\begin{enumerate}
%\pause \item Univariate contrast analysis
%\pause \item Multivariate seachlight analysis
%\pause \item Ridge Regression
%\pause \item LASSO
%\pause \item {\bf SOS LASSO}
%\end{enumerate}
\end{frame}

\section{Simulations}
\subsection{The model; the data}

%%%
%% THE MODEL
%%%
\framedgraphic{A model to simulate our data}{../figures/model_outline.eps}
\framedgraphic{Where information is located}{../figures/between_within_dist.eps}
\framedgraphic{``Localized'' distributed representations}{data_demo_3.pdf}
\framedgraphic{``Dispersed'' distributed representations}{data_demo_4.pdf}

\framedgraphic{Where information is located}{../figures/between_within_dist.eps}

\subsection{Results}
\framedgraphic{Strong localization assumption, within and across}{../figures/univariate.eps}
\framedgraphic{Less strong localization assumptions (but still there)}{../figures/searchlight.eps}
\framedgraphic{No localization assumption (and no feature selection!)}{RidgeTritile_grey.pdf}
\framedgraphic{No localization assumption, with feature selection}{lasso_only.eps}
\framedgraphic{Relaxed localization assumption + feature selection}{soslasso_only.eps}

\begin{frame}
\frametitle{Conclusions}
\begin{enumerate}
\pause \item (Yes, I know I didn't get to the actual brain data.)
\pause \item The assumptions that an method makes about how information is encoded has a large effect on what will be found.
\pause \item Different methods provide different levels of information about the signal it does identify.
\pause \item SOS LASSO appears uniquely suited to test hypotheses about distributed representations in the brain.
\end{enumerate}
\end{frame}

\end{document}

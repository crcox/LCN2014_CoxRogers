\section{Simulation details}
The model shown in Figure \ref{fig.model_outline} was trained on 72 items sampled from two domains, A and B. Each item activated exactly 2 systematic and 2 arbitrary input units, and across items each unit was active in exactly 8 items. Half of the systematic units were activated only by items from domain A, while the remaining half were activated only by items from domain B. Thus any pair of items in the same domain had a small probability of overlapping in some of their systematic properties, while items from different domains never overlapped in their systematic properties. Arbitrary units were equally likely to be active for items from domain A versus B.

The model was fit in LENS \cite{rohde_lens:_1999} using back-propogation to minimize cross-entropy error. The weights were adjusted with a learning rate of 0.1, using momentum (``Doug's'' momentum = 0.9) and subject to weight decay (decay constant = 0.001). The model was trained 10 times to asymptotic performance with very low error over 1000 epochs. Prior to each training run, the model was initialized with random weights sampled from a uniform distribution in the range [-1,1]. These 10 models were used to generate data for 10 model ``subjects,'' based on the patterns of activity elicited by each input over the whole network. Each model was presented with the 72 input patterns in sequence, and the pattern of activation elicited over the 98 units in the network (including the 22 irrelevant units, which always had an activation of zero) was recorded. The dataset for each model subject thus consisted of a matrix with 72 rows corresponding to stimulus items and 98 columns corresponding to model voxels. Each matrix contained the ``true'' response pattern for each subject to each item. To simulate noise in the measurement of this activity, a random value sampled independently from a Gaussian distribution with a mean of zero and standard deviation of 1 was added to each cell of the matrix. We take the resulting values in each cell of a matrix to be a model analog of the estimated BOLD response to a single stimulus at a single voxel in a single subject in an fMRI study. 

To apply different brain-imaging methods to the discovery of structure, it is necessary to further stipulate the anatomical locations of the different units the model. In all simulations, input units were situated all together, with domain-A units neighboring one another, domain-B units neighboring one another, and arbitrary units neighboring one another. Output units were organized the same way, though outputs were assumed to be anatomically distal to inputs. The anatomical arrangement of input and output units was assumed to be identical across model individuals. For hidden units, we considered two different anatomical organizations. For {\em anatomically localized} models, units within a layer (SH, AH, or irrelevant) were also assumed to be anatomical neighbors, localized in the same way across model individuals. In the {\em anatomically dispersed} condition, units from the three hidden layers were assumed to be randomly intermingled with one another anatomically, in a different manner across model individuals. In either case, units in the hidden layers (together with irrelevant units) were assumed to be anatomically distal from both the input and output layers. For each anatomical variant the activation patterns evoked across model units by different inputs, and the ways these patterns were distorted by measurement noise, were identical--all that differed was the assumption about the spatial locations of the units in each layer.


\subsection{Multivariate pattern classification}

The remaining methods we consider are all variants of multi-voxel pattern analysis (MVPA) that rely on pattern classification algorithms \cite{normanMVPA06}. Such approaches reverse the objective underlying univariate analyses: rather than using knowledge of the experimental design to explain variance in neural activity at individual voxels, MVPA uses the variance of neural activity across many voxels to make predictions about the experimental condition to which each trial, stimulus, or time point (henceforth, ``example'') belongs \cite{mitchell_learning_2004, pereira_machine_2009}. To accomplish this, a classification algorithm is applied to a set of {\em training data} which include (a) the pattern of estimated activation evoked over a set of voxels for each of many examples and (b) a set of {\em labels} indicating the experimental condition or class associated with each pattern. For instance,in our model experiment, items in condition A might be labeled with a 0 while items from domain B are labeled with a 1.  From the training data, the algorithm returns a pattern classifier---a statistical model that can be used to predict the label associated with any pattern of activation over voxels. Many different classification algorithms exist in the literature; as just one example, a logistic classifier will return a set of {\em weights}, one for each voxel, such that the estimated voxel activation, multiplied by its weight, summed over all voxels, and subject to a transformation function, yields a number that indicates the pattern label. In our example, a good logistic classifier should yield a number near 1 for all condition A items and near zero for all condition B items.

Even if there is no real signal at all in the data, it may be possible for a classifier to generate correct predictions for all items in the training set, especially when there are many predictors. Training set performance thus does not indicate whether the classifier is exploiting real signal in the data. Instead, the classifier is typically assessed on a {\em hold-out set}: an additional set of examples and labels collected in the same experiment but excluded from the training data. The classifier learned from the training data is applied to patterns in the hold-out set, and for each pattern it generates a ``guess'' about the associated condition label. The classifier output is compared to the true label to get a measure of accuracy. If a model performs above chance at classifying the hold-out set, this indicates that it is likely exploiting real information in the data. To ensure that the results do not depend upon the particular items chosen for the training and hold-out sets, it is common to test a model using {\em n-fold cross-validation}. On each ``fold'' a subset of items is chosen for the hold-out set, and different hold-out sets are selected for different folds, such that, across folds, all items appear in exactly one hold-out set. Each hold-out set provides a measure of model classification accuracy, and this is usually averaged across folds to provide a single number indicating how accurately the trained model can classify hold-out patterns. We will refer to this number as the {\em cross-validation accuracy} of the classifier.

MVPA algorithms, like univariate analyses, are challenged by the abundance of data provided by fMRI, and so must adopt additional assumptions about the nature of the underlying signal. In any fMRI study (as in our model) there will always be more predictors (voxels) than things predicted (stimulus items or events), producing an over-fitting problem. In such cases, there exists no unique solution to the classification problem defined by the training set. Closed-form analyses are undefined, and other model-estimation procedures will produce a classifier that perfectly fits the training data without any guarantee of finding real signal. These problems can only be addressed by constraining the analysis based on an underlying hypothesis about how signal is truly encoded in the data. As with the univariate method, these constraints systematically affect the results. Each of the remaining methods adopt different constraints to solve the over-fitting problem.

\subsubsection{Searchlight concepts and assumptions}
We begin with the well-known ``searchlight'' approach \cite{kriegeskorte_information-based_2006}, which was formulated specifically to address the challenge of finding distributed representations in brain imaging data. The method works as follows. Instead of training a classifier using all predictors at once, a separate classifier is trained for every individual voxel location in every individual subject. For each location, all voxels within a radius $r$ of the center voxel are included as predictors in the classifier. This avoids the over-fitting problem by restricting the number of predictors included in any given classifier.  The mean cross-validation accuracy for each classifier is stored in the searchlight center voxel, providing an {\em information map} for each subject. An univariate group-level analysis can be conducted on the information maps, similar in all respects to the analysis described in the previous section. Per the univariate assumptions, this means that each point in the accuracy map is considered independent of all others. However, within each searchlight, the effect of a given unit on the classification can differ depending upon the activations of other units in the searchlight, and these units can respond to various stimuli in quite different ways. 

Thus the searchlight method relaxes assumptions about the consistency of the neural code within and across individuals, and about the independence of representational units, but retains assumptions about localization of information within and across individuals. These assumptions are summarized in Table \ref{tab.assumptions}, and can be contrasted with those of the univariate method. What results are observed in the model with these differing representational assumptions?

%With this brief overview of the approach, it is useful to consider how the searchlight assumptions about representation compare to those of the univariate method:
%  
%\begin{enumerate}
%\item Localization within individuals: For a classifier to show above-chance cross-validation accuracy, there must be sufficient information contained within its searchlight. If the representation is anatomically distributed such that no searchlight contains sufficient information, the method will not discover the representation. In this sense, the method assumes localization of the representation within individuals, similar to the univariate case.
%
%\item Consistency of coding within individual representations: In contrast to the univariate approach, the method does {\em not} assume any consistency in how information is encoded across voxels within a representation for a given individual. Different voxels contributing to the same representation can respond to various stimuli in quite different ways. So long as the information is present within the searchlight, the classifier can exploit it.
%
%\item Localization across individuals: Like the univariate method, the searchlight approach assumes that the representation will be localized in similar ways across individuals. This assumption licenses the cross-subject test of classifier accuracy at each voxel. If representations are localized differently across individuals, the searchlights that yield above-chance classifications will reside in different anatomical locations in different subjects, so the statistical tests across individuals at common locations will yield null results.
%
%\item Consistency of coding across individuals: In contrast to the univariate case, the approach makes no assumptions about the nature of the code at a given location across individuals. So long as a given location contains information relevant to the classification, the method will detect it, regardless of how the information is coded.
%
%\item Independence of representational units: Unlike the univariate case, the approach does not assume that voxel activations can be interpreted independently. Searchlight classifiers operate on patterns of activation across units in the searchlight, so the effect of a given unit on the classification can differ depending upon the activations of other units in the searchlight. The method does assume, however, that each searchlight can be interpreted independently of every other searchlight. If the information coded in one searchlight varies depending upon the states of units in other searchlights, the method will not discover this.
%\end{enumerate}


\subsubsection{Implementation}
The searchlight analysis was conducted using the SearchMight toolbox \cite{pereira_information_2011} for {\matlab} (Mathworks, 2013a). The input units, hidden units, and output units were treated as three anatomically separated regions so that a searchlight never encompassed units in different regions. This was accomplished by inserting empty units between the layers, and providing a mask to SearchMight to omit those units during analysis while ensuring that no searchlight spans multiple regions. Within each searchlight, a Gaussian Naive Bayes (GNB) classifier was fit to distinguish between category A and B items. Although GNB classifiers are limited in some ways \cite{pereira_information_2011}, the concerns do not apply to this simple and idealized case where noise is truly i.i.d. with uniform variance. The amount of category information in each searchlight was estimated through 6-fold cross validation; the mean cross-validation accuracy was stored at each searchlight center; and the mean accuracy over model subjects was then computed for each unit and tested to see if it differed significantly from chance. The resulting map of p-values is FDR corrected, q<0.05.

As previously, the analysis was performed on both the anatomically localized and the anatomically dispersed arrangements of units.The data were not smoothed prior to the searchlight analysis. The analysis was performed with various searchlight sizes, ranging from 3 to 28. A searchlight size of 7 or 9 should be roughly ``optimal'' given the size of the clusters of informative units in the localized data.

\subsubsection{Results} 

%\begin{center}
%\textbf{---Figure \ref{fig.searchlight} about here---}
%\end{center}

\begin{figure*}[p]
\centering
\includegraphics[width=0.90\textwidth]{figures/figure5.eps}
\caption{Result of the multivariate searchlight analysis of simulated data. Bar height indicates the mean classifier accuracy over subjects. Bars in red indicate searchlights where classification accuracy differed from chance over subjects, p-values corrected to control the false discovery rate at $q<0.05$. Each row shows results from a different searchlight size indicated by the number on the far right. Mean and se indicate the mean and standard error of the classifier accuracy, respectively; n indicates the number of units whose searchlights shows significant information across model subjects.}
\label{fig.searchlight} 
\end{figure*}

Figure \ref{fig.searchlight} shows the results of the searchlight analyses, for localized and dispersed model architectures, and for different searchlight sizes. The format is the same as in the preceding analysis, except the y-axis now indicates the mean classification accuracy for searchlights centered on each unit, rather than a t-value at each unit. As before, colored bars indicate units that the method identifies as statistically significant---that is, units whose surrounding searchlights show classification accuracy reliably above chance across model individuals.

There are several points to note in these plots. First, when the hidden representations are anatomically localized, the method can do a quite good job of identifying both the systematic I/O and the systematic hidden units as important for the domain representations, though for both unit types the results vary substantially with the searchlight size. When the searchlight is small, the method reliably finds the SH units but misses most of the systematic I/O units. This happens because, as noted earlier, the SH units encode a clearer differentiation between domains, so that even if the searchlight does not encompass all 7 units there is sufficient information within it to classify stimuli above chance. The domain distinction is weaker in the systematic-I/O units, so when only a small number of such units fall within the searchlight, there is insufficient information to classify correctly. With a larger searchlight (9 units), the method does very well at finding all relevant units.  When it grows too large, however, it begins to incorrectly flag irrelevant units as being important for the representation (28 units). A very large searchlight, even when centered on an uninformative (arbitrary) unit, can have a broad enough span that it encompasses other informative units. In this case the classifier will perform well by virtue of the informative units appearing in the edge of the searchlight, but the above chance result will be ``stored'' in the searchlight center, making it appear as though there is useful information present at that location. Thus when the signal is anatomically localized, there is a tradeoff between searchlight size and discovery of representational structure, with searchlights that are too small missing weaker signal and those that are too large incorrectly flagging arbitrary signal.

In the model case, where we know {\em a priori} which are the signal-bearing units, it is easy to discern the optimal searchlight size, but it is less clear how this would be determined from real brain imaging data. One might initially expect the optimal searchlight to be identifiable from the accuracy of the resulting classifiers, but Figure \ref{fig.searchlight} suggests that this is not the case: the very large searchlight, which flags many irrelevant units, shows almost as good classifier performance as the optimal size at the SH units and better performance at the systematic I/O units. If we did not already know which units were important for representation in the model, it would be difficult to know which searchlight size to choose, and hence which results to believe.

The second thing to note is that the searchlight analysis does a much poorer job overall of identifying the SH units when these are anatomically dispersed (right panels of Figure \ref{fig.searchlight}). The poor performance arises because the precise anatomical location of the signal-carrying units is assumed to vary across individuals in this case. Within any individual model, a searchlight that includes a few of the informative units will show above-chance performance in classification, but the searchlight centers will differ across model individuals, especially when the searchlights are small. Thus the cross-subject statistical test at each location will yield a null result, leading to poor signal discovery. Larger searchlights will be more likely to contain the signal-carrying units, but also lead to poorer localization of the signal as already noted. 

In sum, the method deals with the over-fitting problem by only including a small number of contiguous voxels in each classifier---an approach which assumes that useful representational structure can be localized within the searchlight radius, in the same locations across subjects. When these assumptions are met, the approach does a good job of discovering representational structure, even if the representational code (i.e., the way that individual units respond to particular stimuli) is highly variable within and across individuals. The limitations noted above arise when the assumptions are violated---when representational structure is anatomically distributed across multiple searchlights (as when searchlights are too small in the localized case), or in different ways across individuals (as in the dispersed model). Moreover, whether the assumptions are met depends, not only upon the anatomical distribution of the signal, but also upon the searchlight size, and it is not clear how the latter can be optimized for real brain imaging data.  

Finally, it is worth noting that, in contrast to the univariate method, the searchlight approach does not provide information about how the contrast of interest in encoded in unit activations. Thus there is no way for the method to show, for instance, that there are some units systematically more active for A items than B items, others showing the reverse pattern, and still others that express the A/B distinction in a distributed code (the SH units). These results are summarized in Table \ref{tab.modelresults}.